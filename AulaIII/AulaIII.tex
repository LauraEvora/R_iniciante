\documentclass{beamer}
%\documentclass[handout,xcolor=pdftex,dvipsnames,table]{beamer}

%\usepackage[latin1]{inputenc}
%\documentclass[ucs]{beamer}%para sistemas com ucs
\usepackage[utf8]{inputenc}
\usepackage{verbatim}
\usepackage{tikz}   %TikZ is required for this to work.  Make sure this exists before the next line
\usepackage{pgf}

 \usepackage[scaled]{helvet}
 \renewcommand*\familydefault{\sfdefault} %% Only if the base font of the document is to be sans serif
 \usepackage[T1]{fontenc}

%\usepackage[utf8x]{inputenc}%idem
\usepackage[brazil]{babel}
\usepackage{verbatim}
\usetheme{Unicamp}%
\usecolortheme{dolphin}
%\usecolortheme{seahorse}
\usefonttheme[onlysmall]{structurebold}

\title[Linguagem R]{R para iniciantes\\ Aula III Leitura e Manipulação de dados}
\author {Carlos Henrique Tonhatti}
%\institute[Unicamp]{Universidade Estadual de Campinas}
\date{}



 \AtBeginSection[]
 {
\begin{frame}
\frametitle{Sumário}
\tableofcontents[currentsection]
\end{frame}
}

\AtBeginSubsection[]
{
   \begin{frame}
       \frametitle{Sumário}

       \tableofcontents[currentsection,currentsubsection]
       \setcounter{tocdepth}{2}
   \end{frame}
}


\begin{document}
%para criar a pagina de rosto
\frame{\titlepage} %inclui a front page 

%==================================================slide
\begin{frame}
  \frametitle{Dúvidas da última Aula?}
\end{frame}

% cria o sumario
\begin{frame}
 \frametitle{Sumário}
 \tableofcontents[pausesections]
  \setcounter{tocdepth}{2}% profundidade do sumario 
\end{frame}
%-------------------------------------------------------
%==================================================slide

%criando um slide


\section{Entrando  e saindo com dados do R}
\begin{frame}{Formas de entrada de dados.}
\centering
   Existem algumas formas de entrar com dados no R: \vspace{10pt}
   \begin{itemize}
   \item Digitando diretamente no terminal (como foi feito nas últimas aulas).
   \item Importando os dados de outras fontes (arquivos, internet).
   \end{itemize}
  \end{frame}
\subsection{Importação de dados}
  \begin{frame}{Importação de dados}
   
   A importação de dados no R depende do tipo e formato dos dados que se pretende importar. A função genérica \texttt{read.table} importa dados em formato de tabela.
   \begin{block}{Importação de dados}
    read.table(file, header = FALSE, sep = "", quote = "\"'",\\
   \hspace{45pt}       dec = ".", row.names, col.names,\\
   \hspace{45pt}       as.is = !stringsAsFactors,\\
   \hspace{45pt}       na.strings = "NA", colClasses = NA, nrows = -1,\\
   \hspace{45pt}       skip = 0, check.names = TRUE, fill = !blank.lines.skip,\\
   \hspace{45pt}             strip.white = FALSE, blank.lines.skip = TRUE,\\
   \hspace{45pt}             comment.char = "\#",\\
   \hspace{45pt}             allowEscapes = FALSE, flush = FALSE,\\
   \hspace{45pt}             stringsAsFactors = default.stringsAsFactors(),\\
   \hspace{45pt}             fileEncoding = "", encoding = "unknown", text)     
   \end{block}
    
  \end{frame}
  \begin{frame}{Exemplo}
Passo 1 - Salvar os dados em txt.
\centering
   \pgfdeclareimage[width=10cm]{rExcel}{rExcel}
    \pgfuseimage{rExcel}
 \end{frame}
 \begin{frame}{Exemplo}
Passo 2 - Salvar o arquivo na pasta de trabalho do R. \pause
\pause
Passo 3 - Importar o arquivo \\
\texttt{> trapa<- read.table(``trapa.txt'',header= TRUE, sep= `` '', dec= ``.'')\\}
Passo 4- Verificar o que foi importado
\texttt{\\
> trapa
\begin{table}[!h] 
\begin{tabular}{cccccc} 
 & codinome & nascimento & estado & vivo & altura \\ 
~[1,] & Didi & 1936 & CE & s & 1.68 \\ 
~[2,] & Dede & 1936 & RJ & s & 1.79 \\ 
~[3,] & Mussum & 1941 & RJ & n & 1.81 \\ 
~[4,] & Zacarias & 1934 & MG & n & 1.61 \\ 
\end{tabular} 
\end{table} 
 }   
 \end{frame}
 \begin{frame}{Considerações}
    \begin{itemize}
 \item O passos do exemplo podem variar devido á complexidade dos
   dados;
 \item O objeto criado é um \texttt{data.frame};
 \item Importante é entender  como os dados estão organizados antes de importar;
 \item  A função \texttt{read.table} é flexível. Pode ser ajustada á vários formatos de dados.\pause   Principais argumentos:
   \begin{description}
   \item [file] nome do arquivo o qual tem os dados,
   \item [header] indica se a primeira linha contem os títulos das colunas,
   \item[sep] indica qual o separador de campos,
   \item[dec] indica qual simbolo é usado para separar decimais.
   \end{description}
\pause
 \texttt{read.table(``trapa.txt'',header= TRUE, sep= `` '', dec= ``.'')}
\pause
Mais informações em \texttt{?read.table}
 \end{itemize}
 \end{frame}
 \begin{frame}{Dicas}
Estabeleça um padrão para os seus dados: \pause
\begin{itemize}
\item Qual separador de campo (TAB, ``,'', ``;'')?
\item Qual simbolo para decimais (``.'', ``,'')?
\item Nome das variáveis na colunas?
\end{itemize}
\pause
Confira se os dados foram lidos de forma correta. Use os comandos file\texttt{ str(), dim(), head(), tail()}
 \end{frame}

\subsection{Exportando dados}
\begin{frame}{Exportando dados do R}
 A exportação de dados do R pode ser feita de várias formas. A função genérica  \texttt{write.table} exporta os dados em forma de \textit{data frame} criando um novo arquivo.
 \begin{block}{Exportação de dados}
 write.table(x, file = `` '', append = FALSE, quote = TRUE, sep = `` '',\\
     \hspace{45pt}  eol = ``\textbackslash n'', na = ``NA'', dec = ``.'', row.names = TRUE,\\
     \hspace{45pt}    col.names = TRUE, qmethod = c(``escape'', ``double''),\\
      \hspace{45pt}            fileEncoding = `` '')   
 \end{block}
\texttt{> write.table(trapa, file=``trapa2.txt'')}
  
\end{frame}

\section{Manipulando dados}
\subsection{Edição}
\begin{frame}{Edição}
 A função \texttt{fix()} abre o objeto dentro de uma planilha possibilitando pequenas edições. \\
\centering
\pgfdeclareimage[width=5cm]{fix}{fix}
\pgfuseimage{fix}  
\end{frame}


\begin{frame}{Edição: nomes de colunas e linhas}

\begin{block}{Nomes das colunas e linhas}
\# Retorna o nome das colunas\\
names(x)\\
colnames(x)\\
\# Retorna o nome das linhas\\
rownames(x)\\
 \end{block}
\end{frame}

\begin{frame}{Exemplo}
\texttt{> names(trapa)\\
~[1]  ``codinome''   ``nascimento'' ``estado''   ``vivo''  ``altura''}\vspace{10pt}

É possível alterar o nome das colunas ou linhas usando a mesma função:\\ \vspace{10pt}

\texttt{> names(trapa)<-c(``COD'', ``NASC'', ``UF'', ``VIVO'', ``Altura'')}

\begin{table}[!h]  
 \begin{tabular}{ cccccc} 
 & COD & NASC & UF & VIVO & Altura \\ 
~[1,] & Didi & 1936 & CE & s & 1.68 \\ 
~[2,] & Dede & 1936 & RJ & s & 1.79 \\ 
~[3,] & Mussum & 1941 & RJ & n & 1.81 \\ 
~[4,] & Zacarias & 1934 & MG & n & 1.61 \\ 
\end{tabular} 
\end{table}
\end{frame}

\subsection{Criação de subconjuntos}
\begin{frame}{Subconjuntos}
\centering
\pgfdeclareimage[width=10cm]{subconjunto}{subconjunto}
\pgfuseimage{subconjunto}
\pause
Sinônimo de indexação.  
\end{frame}

\begin{frame}{Subconjuntos: vetores}

  \begin{center}
    \pgfdeclareimage[width=6cm]{vetor}{vetor} \pgfuseimage{vetor}
  \end{center}


\texttt{> animal<-c( ``peixe'' ,``cão'',``gato'',``hamster'')}  
\pause

\texttt{> animal[2]\\
~[1] ``cão''\\ \pause
> animal[c(2,3)]\\
~[1] ``cão''  ``gato''\\
> animal[2:4]\\
~[1] ``cão''     ``gato''    ``hamster''}
\end{frame}

\begin{frame}{Subconjuntos: vetores}
\texttt{> animal<-c( ``peixe'' ,``cão'',``gato'',``hamster'')\\}  
\pause
\texttt{> animal[c(1,1,2,3)]\\
~[1] ``peixe'' ``peixe'' ``cão''   ``gato'' \\ \pause
> animal[-2]\\
~[1] ``peixe''   ``gato''  ``hamster''\\
  }
\end{frame}

\begin{frame}{Subconjuntos: matrizes e \textit{data.frame}}
Matrizes e \textit{data.frames} possuem 2 dimensões.

\begin{block}{Subconjunto em uma matriz}
matriz[linha, coluna]  
\end{block}
\texttt{> x<-matrix(1:12,ncol=3)\\
> x}
\begin{table}[h]  
 \begin{tabular}{cccc} 
 & [,1] &[,2]&[,3] \\
~[1,]& 1& 5& 9 \\
~[2,] &   2 &   6 &  10\\
~[3,] &   3  &  7 &  11\\
~[4,] &   4  &  8 &  12\\
\end{tabular} 
\end{table} 

\texttt{> x[2,2]\\
~[1] 6}
\end{frame}

\begin{frame}{Subconjuntos: matrizes e \textit{data.frame}}
\texttt{
  > x[2:4,2:3]\\
\begin{table}[h]  
 \begin{tabular}{cccc} 
    & [,1]& [,2]\\
~[1,]  &  6  & 10\\
~[2,]  &  7 &  11\\
~[3,]  &  8&   12\\
\end{tabular} 
\end{table}} \pause
Todas as colunas de uma linha\\
\texttt{> x[1,]\\
~[1] 1 5 9}\\ \pause
Todas as linhas de uma coluna\\
\texttt{> x[,1]\\
~[1] 1 2 3 4}

\end{frame}





\begin{frame}{Subconjunto: nomes das colunas em \textit{data.frame}}
  É possível usar o nome da coluna de  um \texttt{data.frame}. 
  \begin{block}{Seleção do  conteúdo de uma coluna}
   objeto\textbf{\$}coluna\\   
  \end{block}
\texttt{> trapa\$COD\\
~[1] Didi     Dede     Mussum   Zacarias\\
Levels: Dede Didi Mussum Zacarias\\
> trapa\$NASC\\
~[1] 1936 1936 1941 1934\\}
  \end{frame}

  \begin{frame}{Subconjunto: nomes das colunas em \textit{data.frame}}
É possível alterar o conteúdo de uma coluna

\texttt{> trapa\$VIVO<-c(``T'', ``T'', ``F'', ``F'')} \pause \vspace{10pt}

E criar novas colunas

\texttt{> trapa\$idade<- 2014-trapa\$NASC}

\texttt{\begin{table}[!h]
\begin{tabular}{ ccccccc} 
  & COD & NASC & UF & VIVO & Altura & idade \\ 
~[1,] & Didi & 1936 & CE & T & 1.68 & 78 \\ 
~[2,] & Dede & 1936 & RJ & T & 1.79 & 78 \\ 
~[3,] & Mussum & 1941 & RJ & F & 1.81 & 73 \\ 
~[4,] & Zacarias & 1934 & MG & F & 1.61 & 80 \\ 
\end{tabular} 
\end{table}}
  \end{frame}


\begin{frame}{Subconjuntos: listas}
\texttt{> aluno<-list(nome= ``José'', idade=20, esportes=c(``natação'', ``surfe''))\\ \pause
> aluno\\
\$nome\\
~[1] ``José''\\
\vspace{10pt}
\$idade\\
~[1] 20\\
\vspace{10pt}
\$esportes\\
~[1] ``natação'' ``surfe''}
\end{frame}

\begin{frame}{Subconjuntos: listas}
\texttt{> aluno\$esportes\\
~[1] ``natação'' ``surfe'' \\ \vspace{10pt} \pause
> aluno[3]\\
\$esportes\\
~[1] ``natação'' ``surfe'' \\ \vspace{10pt} \pause
> aluno[[3]]\\
~[1] ``natação'' ``surfe'' }

\end{frame}

\begin{frame}{Subconjuntos: listas}
\texttt{> aluno\$esportes\\
~[1] ``natação'' ``surfe'' \\ \vspace{10pt} \pause
> aluno\$esportes[1]\\
~[1] ``natação''\\ \vspace{10pt} \pause
> aluno[[3]][1]\\
~[1] ``natação''}  
\end{frame}
\subsection{Criação de filtros}
\begin{frame}{Usando filtros}
Um filtro é uma condição lógica que permite selecionar dados.\\
Para isso é usado os operadores comparação e lógicos. \\
\begin{block}{Operadores lógicos}
\begin{table}[!h]
\begin{tabular}{cc} 
\& \&\&& E \\
| || & OU \\
!& Negação \\
\end{tabular} 
\end{table}  
\end{block}  
\end{frame}

\begin{frame}{Exemplos}
\texttt{> z<-c(1,2,3,4,5,6,7,8)\\ \vspace{10pt}
> z>3\\
~[1] FALSE FALSE FALSE  TRUE  TRUE  TRUE  TRUE  TRUE \\ \vspace{10pt}
> z[z>3]\\
~[1] 4 5 6 7 8 \\ \vspace{10pt} \pause
> z>3 \& z<7\\
~[1] FALSE FALSE FALSE  TRUE  TRUE  TRUE FALSE FALSE\\
> z[z>3 \& z<7] \\
~[1] 4 5 6
}
\end{frame}
\begin{frame}{Exemplos}

\texttt{> Altura\\
~[1] 1.85 1.78 1.92 1.63 1.81 1.55\\
> Sexo\\
~[1] M M M F F F\\
Levels: F M\\ \vspace{10pt} \pause
> homens.altos<- Altura >= 1.80 \& Sexo== ``M''\\
> homens.altos\\
~[1]  TRUE FALSE  TRUE FALSE FALSE FALSE}
 \end{frame}

 \begin{frame}{Exemplos}
\texttt{ > homens.altos \\
~[1]  TRUE FALSE  TRUE FALSE FALSE FALSE \\ \vspace{10pt}}
Quantos homens altos existem?\\
\texttt{> sum(homens.altos)\\
~[1] 2\\}
Proporção de homens altos \\
\texttt{> mean(homens.altos)\\
~[1] 0.3333333\\}
 \end{frame}

 \begin{frame}{Exemplos}
 \texttt{> ilhas}\\
 \begin{table}[!h]
\begin{tabular}{ccccc} 
   &[,1]&[,2]&[,3]&[,4]\\
~[1,]  &  3  &  2 &   0  &  2\\
~[2,]  &  1  &  0 &   0  &  2\\
~[3,]  &  1  &  1 &   1  &  0\\
~[4,]  &  1  &  2 &   2  &  0\\
~[5,]  &  1  &  1 &   2  &  2\\
 \end{tabular} 
\end{table}

\texttt{> ilhas>1 \\}
 \begin{table}[!h]
\begin{tabular}{ccccc}
     & [,1]&  [,2]&  [,3] & [,4]\\
~[1,]&  TRUE & TRUE & FALSE & TRUE\\
~[2,]& FALSE &FALSE & FALSE & TRUE\\
~[3,]& FALSE &FALSE & FALSE & FALSE\\
~[4,]& FALSE & TRUE & TRUE & FALSE\\
~[5,]& FALSE &FALSE & TRUE & TRUE\\
 \end{tabular} 
\end{table}
 \end{frame}

 \begin{frame}{Exemplos}
\texttt{ > apply(ilhas>1,2,sum)\\
~[1] 1 2 2 3}
 \end{frame}

\subsection{Ordenação}
\begin{frame}{Ordenação}
Ordenação é o ato de se colocar os elementos de uma sequência de informações, ou dados, em uma ordem predefinida.

\begin{block}{Funções de ordenação}
\# Ordena o vetor\\
sort(x, decreasing=FALSE) \\ \vspace{10pt}
\# Retorna os indices ordenados \\
order(x, decreasing=FALSE) \\ \vspace{10pt}
\end{block}
  
\end{frame}

\begin{frame}{Exemplos}
\texttt{> sort(Altura)\\
~[1] 1.55 1.63 1.78 1.81 1.85 1.92\\ \vspace{10pt} \pause
> order(Altura)\\
~[1] 6 4 2 5 1 3 \\ \vspace{10pt} \pause
> Altura[order(Altura)]\\
~[1] 1.55 1.63 1.78 1.81 1.85 1.92\\
>Altura.Crescente<-Altura[order(Altura)]}  
\end{frame}
\begin{frame}{Cuidado ao ordenar matrizes/\textit{data.frames}}
\centering
\pgfuseimage{subconjunto}  
\end{frame}

\begin{frame}{Mantendo a relação entre as colunas}
\texttt{> order(trapa\$COD)\\
~[1] 2 1 3 4\\ \vspace{10pt}
> Em.ordem<-order(trapa\$COD)\\ \pause
> trapa.em.ordem<-trapa[Em.ordem,]\\
> trapa.em.ordem}
 \begin{table}[!h]
\begin{tabular}{cccccc}
      & COD & NASC & UF & VIVO & Altura\\
2 &    Dede & 1936 &  RJ&    s &  1.79\\
1 &    Didi & 1936 & CE &   s &  1.68\\
3 &  Mussum & 1941 & RJ &   n  & 1.81\\
4 &Zacarias  & 1934 & MG &    n &  1.61\\
\end{tabular}
\end{table}
  \end{frame}

  \begin{frame}{Próxima aula}
    \begin{itemize}
    \item Ler cap. 4 da apostila.
    \item  Fazer tutorial ``Leitura e manipulação de dados''.
    \item  Fazer exercícios para entregar.
    \end{itemize}
  \end{frame}

\end{document}
  
