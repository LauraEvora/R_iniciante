\documentclass[handout]{beamer}
%\documentclass[handout,xcolor=pdftex,dvipsnames,table]{beamer}

%\usepackage[latin1]{inputenc}
%\documentclass[ucs]{beamer}%para sistemas com ucs
\usepackage[utf8]{inputenc}
\usepackage{verbatim}
\usepackage{tikz}   %TikZ is required for this to work.  Make sure this exists before the next line
\usepackage{pgf}

 \usepackage[scaled]{helvet}
 \renewcommand*\familydefault{\sfdefault} %% Only if the base font of the document is to be sans serif
 \usepackage[T1]{fontenc}

%\usepackage[utf8x]{inputenc}%idem
\usepackage[brazil]{babel}
\usepackage{verbatim}
\usetheme{Unicamp}%
\usecolortheme{dolphin}
%\usecolortheme{seahorse}
\usefonttheme[onlysmall]{structurebold}

\title[Linguagem R]{R para iniciantes\\ Aula I \\ Programação}
\author {Carlos Henrique Tonhatti}
%\institute[Unicamp]{Universidade Estadual de Campinas}
\date{}



 \AtBeginSection[]
 {
\begin{frame}
\frametitle{Sumário}
\tableofcontents[currentsection]
\end{frame}
}

\AtBeginSubsection[]
{
   \begin{frame}
       \frametitle{Sumário}

       \tableofcontents[currentsection,currentsubsection]
       \setcounter{tocdepth}{2}
   \end{frame}
}


\begin{document}
%para criar a pagina de rosto
\frame{\titlepage} %inclui a front page 

%==================================================slide

% cria o sumario
\begin{frame}
 \frametitle{Sumário}
 \tableofcontents[pausesections]
  \setcounter{tocdepth}{2}% profundidade do sumario 
\end{frame}
%-------------------------------------------------------
%==================================================slide

\section{Apresentação da disciplina}
\begin{frame}
  \begin{center}
  \frametitle{Objetivo da disciplina}

  \begin{itemize}
  \item Apresentar a linguagem R. 
  \item Introduzir noções básicas de programação.
  \item Dar autonomia ao aluno para solucionar problemas programando em R. 
 \end{itemize}
\end{center}
\end{frame}


\begin{frame}
\begin{center}
  \frametitle{Estrutura da disciplina}
  \begin{itemize}
  \item Aulas presencias no período da manhã.
  \item Plantões  facultativos no  período da tarde.
  \item Um tutorial e um exercício prático para cada aula.
  \item Tarefas para serem entregues.
  \end{itemize}
\end{center}
\end{frame}



\begin{frame}
  \begin{center}
  \frametitle{Tutoriais, exercícios e tarefas}
  \begin{itemize}
  \item Material base \url{http://ecologia.ib.usp.br/bie5782/doku.php}
  \item Material dessa disciplina \url{https://tomatebio.github.io/R_iniciante/}
  \end{itemize}
\end{center}
\end{frame}





\section{Programação}
\begin{frame}{Programar é preciso}

  \begin{quote}
    Uma das coisas mais importantes que você pode fazer é dedicar um
    tempo para aprender uma linguagem de programação de verdade.

    Aprender a programar é como aprender outro idioma: exige tempo e
    treinamento, e não há resultados práticos imediatos. Mas se você
    supera essa primeira subida íngreme da curva de aprendizado, os
    ganhos como cientista são enormes. 

    Programar não vai apenas
    livrar você da camisa de força dos pacotes estatísticos, mas
    também irá aguçar suas habilidades analíticas e ampliar os
    horizontes de modelagem ecológica e estatística.
  \end{quote}

  Tradução um tanto livre de Gotelli \& Ellison,
2004. A Primer of Ecological Statistics. Sunderland, Sinauer.
  
\end{frame}

\begin{frame}{Todos podem programar}
\centering
\pgfdeclareimage[width=10cm]{code}{code}
\pgfuseimage{code} \\
\url{https://hourofcode.com/br}
\end{frame}

\begin{frame}{Alguns termos}
   \begin{description}[<+->]
   \item[Programa] Sequência de instruções escritas para realizar uma tarefa especifica;
    \item[Linguagem de programação] é um método padronizado para comunicar instruções para um computador. É um conjunto de regras sintáticas e semânticas usadas para definir um programa de computador;
    \item[Código fonte] é o conjunto de palavras ou símbolos escritos de forma ordenada, contendo instruções em uma das linguagens de programação existentes, de maneira lógica;
     \item[``Máquina''] Equipamento eletrônico: computador, servidor, celular, urna eletrônica, etc.
   \end{description} 
  \end{frame}

\begin{frame}{Linguagens de programação}
  \begin{block}{Linguagem (Código) de máquina}
     Conjunto de instruções em código binário que são compreendidos pela CPU. Dependente do tipo de máquina.\\
\texttt{8B542408 83FA0077 06B80000 0000C383\\
FA027706 B8010000 00C353BB 01000000}
  \end{block}
\pause
  \begin{block}{Linguagem de baixo nível}
        Linguagem de programação  dependente do tipo de máquina de fácil tradução para a máquina. Ex. \textit{assembly}\\
fib: \\
\hspace{30pt}    mov edx, [esp+8]\\
\hspace{30pt}    cmp edx, 0\\
\hspace{30pt}    ja @f\\
\hspace{30pt}    mov eax, 0\\
\hspace{30pt}    ret\\    
  \end{block}
  \end{frame}

  \begin{frame}{Linguagens de programação}
    \begin{block}{Linguagens de alto nível}
         Linguagem de programação independente do tipo de máquina e de fácil utilização pelo ser humano. Ex. Pascal, C, Algol, Fortran, java, R, Python, Perl, etc\\
 \texttt{unsigned int fib(unsigned int n)\\
 \{\\
\hspace{10pt}    if (n <= 0)\\
\hspace{10pt}\hspace{10pt}        return 0;\\
\hspace{10pt}    else if (n <= 2)\\
\hspace{10pt}\hspace{10pt}        return 1;\\
\hspace{10pt}    else \{\\
\hspace{10pt}\hspace{10pt}        int a,b,c;\\
\hspace{10pt}\hspace{10pt}        a = 1;\\
\hspace{10pt}\hspace{10pt}        b = 1;\\
\ldots\\
  } 
    \end{block}
  \end{frame}


  \begin{frame}{Tradução das linguagens}
    \begin{itemize}
    \item  Linguagens de alto ou baixo nível precisam ser
      traduzidas para código de máquina para que o computador possa
      executar. \pause
    \item O modo de tradução depende do tipo de linguagem.
      \begin{itemize}
      \item Compilação.
      \item Interpretação.
      \end{itemize}
    \end{itemize}
\pause
    \begin{center}
      \pgfdeclareimage[width=6cm]{tradutores}{tradutores}
      \pgfuseimage{tradutores}
    \end{center}
  \end{frame}

  \begin{frame}{Compilação $\times$ interpretação}
\centering
        \pgfdeclareimage[width=10cm]{compilar}{compilar}
      \pgfuseimage{compilar}\\ \pause
  A forma de tradução é uma característica da linguagem.
  \end{frame}


\subsection{Algoritmos}
\begin{frame}{Definição}

  \begin{block}{Algoritmo}
    Uma sequência finita de instruções  bem definidas e não ambíguas, e cada uma das quais pode ser executada mecanicamente num período de tempo finito e com uma quantidade de esforço finita.
  \end{block} 
  \begin{itemize}[<+->]
  \item Não representa necessariamente um programa de computador;

  \item Diferentes algoritmos podem realizar a mesma tarefa usando um
    conjunto diferenciado de instruções em mais ou menos tempo, espaço
    ou esforço que outros.
  \end{itemize}
\end{frame}

\begin{frame}{Representação visual}
Um algoritmo (ou mesmo um programa inteiro) pode ser representado na forma de um fluxograma.\\

\begin{center}
  \pgfdeclareimage[width=3cm]{fluxograma}{fluxograma}
  \pgfuseimage{fluxograma}
\end{center}
\end{frame}

\begin{frame}{Exemplo}
\begin{center}
  \pgfdeclareimage[width=9cm]{bigbang}{bigbang}
  \pgfuseimage{bigbang}
\end{center}
\end{frame}

\begin{frame}{Representação por texto}
  Um algoritmo (ou mesmo um programa inteiro) pode ser escrita de forma informal na forma de ``pseudocódigo''\\
\texttt{INÍCIO\\
\hspace{10pt}   Abrir o Forno\\
\hspace{10pt}   SE(Forno== acesso)\\
\hspace{10pt}\hspace{10pt}       Botar lenha\\
\hspace{10pt}   CASO CONTRÁRIO\\
\hspace{10pt}\hspace{10pt}       Acender fogo\\
\hspace{10pt}   Assar pão\\
FIM\\}
Vantagem de não ser dependente de uma linguagem. Pode ser traduzida para qualquer linguagem de programação.
\end{frame}

\begin{frame}{Recomendações}
   Quando for começar um novo trabalho:
  \begin{itemize}[<+->]
  \item Escreva todos os passos em ``pseudocódigo'' ou desenhe um fluxograma. Isso vai ajudar a entender todos os passos,
  \item Simule os passos no papel. Veja se não ficou nada ambíguo,
  \item Implemente em uma  linguagem de programação por partes e teste cada parte.
  \item Use comentários!
  \end{itemize}
  \pause
Para ver como se programa em outras linguagens \url{http://rosettacode.org/wiki/Rosetta_Code}
\end{frame}


 \section{Linguagem  R}
\subsection{Caracteristicas}

 \begin{frame}
    \frametitle{O que é R}
   \begin{center}

   \begin{itemize}
   \item Ambiente de programação: coerente e integrado. 
   \item Conjunto de ferramentas para manipulação de dados, cálculos e apresentação gráfica.
   \item Linguagem S (Bell Labs).
   \end{itemize}
 \end{center}
 \end{frame}

 \begin{frame}
  \frametitle{Características do R }
  \begin{columns}
    \begin{column}{6cm}
      \begin{center}
        \begin{itemize}
        \item Código aberto: projeto colaborativo.
        \item Flexibilidade.
        \item Disponível para Windows, Mac e Linux.
        \item É orientado a objetos.
        \item Interface com outras linguagens.
        \end{itemize}
      \end{center}
    \end{column}
    \begin{column}{6cm}
    \pgfdeclareimage[width=6cm]{R}{R}
  \pgfuseimage{R}
    \end{column}
  \end{columns}

 \end{frame}


 \subsection{Estrutura do R}
 \begin{frame}
   \frametitle{Avaliador de expressões}
   \begin{center}
      O R avalia as expressões digitadas:
\pause
      \[ Ler \rightarrow Analisar ~ ~(sintaxe)  \rightarrow Avaliar \]
\pause
     \end{center}
     O R lê o que foi escrito pelo usuário, analisa o texto como uma expressão e tenta avaliar esta expressão. Normalmente retorna uma resposta na tela.
 
 
 \end{frame}
 \begin{frame}
   \frametitle{Exemplos}
          \texttt{
            > 12,3 \\
            Erro: ',' inesperado in "12," \\
\pause
            > oi \\
            Erro: objeto 'oi' não encontrado \\
\pause
             > 1 \\
            ~[1] 1 \\ 
\pause
            > a=1 \\
            >}  
 \end{frame}

 \begin{frame}
   \frametitle{Sintaxe básica}
   \begin{block}{Sintaxe básica de atribuição}
     \# Mais usado. \\
     objeto \textless- expressão  \\ \vspace{10pt}
    \# Não significa igualdade.\\
    objeto = expressão \\ \vspace{10pt} 
    \# Pouco usado.\\
    expressão -\textgreater objeto \hspace{10pt}        
   \end{block}
\pause
    \texttt{> a<-4\\
           > a=4 \\
           > 4->a\\}
 \pause
\vspace{8pt}
 Digite o nome para exibir.

      \texttt{> a \\
              ~[1] 4}
 \end{frame}

 \begin{frame}
 \frametitle{Sintaxe básica}
 \begin{block}{Sintaxe básica das funções}
  função(argumento1=valor,argumento2= valor,...)
 \end{block}
 \texttt{> plot(x=area,y=peso)\\
        > plot(y=peso,x=area)\\
        > plot(area,peso)}

         \vspace{10pt} 
\pause
         É diferente de: \\
       \texttt{ > plot(peso,area)} \\
\pause
\vspace{10pt}
O R interpreta os argumentos pelo nome ou pela posição dentro da expressão. Não é preciso colocar o nome do argumento se a posição for mantida. 
 \end{frame}
 \begin{frame}
   \frametitle{Não esqueça dos parênteses}
   Mesmo sem argumentos é necessário digitar os parênteses. 
   \vspace{10pt}

\texttt{\small  > citation()\\
     \vspace{10pt}
    To cite R in publications use:\\ 
    \vspace{10pt}
  R Core Team (2013). R: A language and environment for statistical 
  computing. R Foundation for Statistical Computing, Vienna, Austria. \\
\pause
\vspace{10pt}
 > citation \\
 function (package =  "base", lib.loc = NULL, auto = NULL) \\
\{ \\
    if (inherits(auto, "packageDescription")) \{ \\
    \vdots   
  } 
 \end{frame}
 \begin{frame}
   \frametitle{Sabendo os argumentos de uma função}
   \begin{block}{Retorna os argumentos de uma função}
      args(Nome da função) \\
       \end{block}
\pause
\texttt{> args(ls) \\
function (name, pos = -1L, envir = as.environment(pos), all.names = FALSE, 
    pattern) \\
NULL\\}
\pause
\vspace{10pt}
   Cartão de referência para imprimir que contêm as funções mais usadas.

 Disponível em \url{http://cran.r-project.org/doc/contrib/Short-refcard.pdf}  

\end{frame}
\begin{frame}
  \frametitle{Ajuda sobre funções}
   Felizmente existem várias formas de buscar ajuda sobre as funções.
   \begin{block}{Funções de ajuda}
      \# Abre texto de ajuda. \\
      help(Nome da função)\\
     ?Nome da função    \\
     \vspace{10pt}
      \# Abre a ajuda no navegador.\\
       help.start() \\
     \vspace{10pt}
      \# Busca dentro dos arquivos de ajuda.\\
        help.search(``palavra-chave'')\\
        ? ? ``palavra-chave'' \\
       apropos(``palavra-chave'')
   \end{block}
\end{frame}
\begin{frame}
  \frametitle{Ajuda online}
  \begin{description}[<+->]
    \item[Google] Não funciona muito bem, ``R'' é uma palavra fácil de confundir. Sugestão:\\ \texttt{filetype:R ``palavra-chave'' -rebol}
      \item[Rseek] Buscador especifico para R. Funciona muito bem. \url{www.rseek.org}
      \item[Site do R] Os manuais oficiais do R, material produzido por usuários, listas de email.\\
     \url{http://cran.r-project.org/}  
  \end{description}
\end{frame}
\begin{frame}
  \frametitle{Outros endereços}
  \begin{description}[<+->]
     \item[R-bloggers] Noticias, Tutoriais enviados por usuários.\\
           \url{http://www.r-bloggers.com}
     \item[Quick-R] Tutoriais e exemplos curtos.\\
          \url{http://www.statmethods.net/}
    \item[Apostila online] Apostila usada no curso da Ecologia da USP. \\
         \url{http://ecologia.ib.usp.br/bie5782}
    \end{description}
\end{frame}

\subsection{Linguagem Orientada a Objetos}

\begin{frame}
  \frametitle{Linguagem Orientada a Objetos}
  \begin{block}{Objeto}
    \begin{quote}
      It is a location in memory having a \textcolor{red}{value} and referenced by an \textcolor{red}{identifier}. An object can be a variable, function, or data structure.
      \end{quote}
\url{http://en.wikipedia.org/wiki/Object_(computer_science)} 
\end{block}
\pause
\texttt{> a<-42}
\end{frame}

\begin{frame}
  \frametitle{Objetos pertencem a classes}
  \begin{block}{Classe}
    \begin{quote}
      It is an extensible \textcolor{red}{template for creating
        objects}, providing initial values for state (member
      variables) and implementations of \textcolor{red}{behavior} (member functions, methods)
    \end{quote}
\url{http://en.wikipedia.org/wiki/Class_(computer_science)}

  \end{block}
\end{frame}

\begin{frame}
  \frametitle{Principais classes de objetos no R}
  \begin{itemize}
    \item Vetores.
    \item Fatores.
    \item Matriz.
    \item Lista.
    \item \textit{Data frame}.
    
  \end{itemize}
  \end{frame}

  \begin{frame}
    \frametitle{Vetor}
    \begin{block}{\textit{Vector}}
     É um conjunto de elementos de uma mesma classe
      (números, caracteres, lógicos, etc) organizados em uma
      dimensão. Não corresponde aos vetores da álgebra linear.
    \end{block}
\vspace{10pt}
\texttt{a<-4\\
> Animal<-c(``Cachorro'', ``Gato'', ``Mosquito'')\\
> Peso<-c(1500,1000,1)\\
> Pet<-c(TRUE,TRUE,FALSE)\\}
  \end{frame}
  \begin{frame}
    \frametitle{Fatores}
    \begin{block}{\textit{Factor}}
     Semelhante aos vetores de R. Os elementos são divididos  em níveis  \textit{levels}. Usado para dados categóricos.      
    \end{block} 
\vspace{10pt}
\texttt{> doce <-factor(c("light","light","diet","diet")) \\
> doce \\
~[1] light light diet  diet \\
Levels: diet light}
  \end{frame}
  \begin{frame}
    \frametitle{Fatores}
    \begin{block}{Níveis de um fator}
        \# Retorna os níveis de um fator. \\
        levels(Nome do fator)
    \end{block}
\vspace{10pt}
\texttt{> levels(doce)\\
~[1] ``diet''   ``light''}\\
\vspace{10pt}
Adicionando um nível que não há na amostra \\ 
\texttt{> levels(doce)<-c("diet","light","açúcar")\\
> doce \\
~[1] light light diet  diet \\
Levels: diet light açúcar}
  \end{frame}
  \begin{frame}
    \frametitle{Matriz}
    \begin{block}{\textit{Matrix}}
     É um vetor com dois atributos adicionais: Número de linhas e colunas. Corresponde a matriz da Álgebra linear. 
    \end{block}
\[A_{2,3}= \begin{pmatrix}
     1 &4 &5 \\
     3 & 6& 7
   \end{pmatrix} \]
\texttt{> A <-matrix(c(1,4,5,3,6,7),nrow=2,ncol=3)\\
> A <-matrix(c(1,4,5,3,6,7),nrow=2)\\
> A\\
  \hspace{20pt} ~[,1]~[,2]~[,3]\\
~[1,]  ~  1  ~  5 ~   6\\
~[2,]  ~  4 ~   3  ~  7
}
  \end{frame}

  \begin{frame}
    \frametitle{Lista}
    \begin{block}{\textit{List}}
        É uma coleção de objetos de diferentes classes. Pode conter outras listas.     
    \end{block}
\texttt{> lista.qualquer<-list(doce,A)\\
  > lista.qualquer \\
 ~[[1]]\\
~[1] light light diet  diet \\
Levels: diet light açúcar\\
\vspace{10pt}
[[2]]\\
  \hspace{20pt} ~[,1]~[,2]~[,3]\\
~[1,]  ~  1  ~  5 ~   6\\
~[2,]  ~  4 ~   3  ~  7}
  \end{frame}
  \begin{frame}
    \frametitle{Data frame}
    \begin{block}{\textit{Data frame}}
     É um tipo especial de lista no qual os componentes tem o mesmo tamanho. Semelhante às planilhas de dados do excel.      
    \end{block}

\texttt{> Animal<-c(``Cachorro'', ``Gato'', ``Mosquito'')\\
>  Peso <-c(1500,1000,1)\\
>  Pet <-c(TRUE,TRUE,FALSE)\\
> Bichos <-data.frame(Animal,Peso,Pet)\\
> Bichos\\
\begin{tabular}{r r r r}
   &  Animal& Peso&   Pet  \\
1 & Cahorro& 1500&  TRUE \\
2 &    Gato& 1000&  TRUE \\
3 & Mosquito &   1& FALSE \\
\end{tabular}
}
  \end{frame}
  \begin{frame}
    \frametitle{Sobre as classes}
    \begin{block}{Identificando a classe de um objeto}
      \# Retorna qual a classe de um objeto.\\
      class(Nome do objeto)\\ \vspace{10pt}
      \# Retorna \texttt{Verdadeiro} ou \texttt{Falso} para uma determinada classe. \\
       is.vector(Nome do objeto)\\
       is.factor(Nome do objeto)\\
        \vdots 
         \vspace{10pt}
        \# Retorna a estrutura de um objeto.\\
         str(Nome do objeto)
    \end{block}
  \end{frame}
  \begin{frame}

\texttt{\small > class(Bichos)\\
~[1] "data.frame"\\
> is.vector(Bichos) \\
~[1] FALSE \\
> is.data.frame(Bichos)\\
~[1] TRUE \\
\vspace{10pt}
> str(Bichos)\\
'data.frame':	3 obs. of  3 variables: \\
\$ Animal: Factor w/ 3 levels "Cahorro","Gato",..: 1 2 3\\
\$ Peso ~ : num  1500 1000 1\\
\$ Pet ~  : logi  TRUE TRUE FALSE}
  \end{frame}
  \begin{frame}
    \frametitle{Listar e Remover}
    \begin{block}{Listar e remover}
     \# Lista todos os objetos criados. \\ 
     ls()\\
      \# Remove um objeto. Sem recuperação. \\
      rm(Nome do Objeto)
    \end{block}
   \texttt{> ls()\\
           > rm(Bichos)
}
  \end{frame}
\subsection{Pacotes}
  \begin{frame}
    \frametitle{Pacotes}
    Todas as funções do R são organizadas em pacotes. O pacote \texttt{base} é instalado junto com o R e traz as funções mais básicas.\\
    \vspace{10pt}
 Pacotes adicionais estão disponíveis na Internet. Cada pacote adiciona novas funções e métodos e, normalmente são feitos para uma tarefa especifica.\\
   
  \end{frame}

  \begin{frame}
    \frametitle{Achando um pacote}
    \begin{center}
      Há mais de 12000 pacotes no site oficial do R.
\vspace{10pt}
\url{https://cloud.r-project.org/}  
    
      \begin{columns}
        \begin{column}{6cm}
          \begin{center}
           Índice alfabético

            \pgfdeclareimage[width=5cm]{pacotesNome}{pacotesNome}
            \pgfuseimage{pacotesNome}           
          \end{center}

        \end{column}
        \begin{column}{6cm}
          \begin{center}
           Organizados por tarefa

            \pgfdeclareimage[width=5cm]{taskViews}{taskViews}
            \pgfuseimage{taskViews}
          
          \end{center}
        \end{column}
      \end{columns}

\end{center} 
  \end{frame}
  \begin{frame}
    \frametitle{Instalando um novo pacote}
    \begin{block}{Instalar um novo pacote}
       install.packages(``Nome do pacote'')
    \end{block}
   É necessário carregar o pacote antes de usar as funções.
   \begin{block}{Carregando um pacote}
     library(Nome do pacote)\\
     require(Nome do pacote)

   \end{block}
  \end{frame}
\section{Sessão de trabalho no R}
\begin{frame}
  \frametitle{Área de trabalho}
  \begin{center}
          \pgfdeclareimage[width=6cm]{bancada}{bancada}
            \pgfuseimage{bancada}
  \end{center}
\end{frame}

\begin{frame}
  \frametitle{\textit{Workspace}}
  \begin{center}
    \pgfdeclareimage[width=12cm]{workspace}{workspace}
    \pgfuseimage{workspace}
  \end{center}
\end{frame}
\begin{frame}
  \frametitle{Iniciar uma sessão de trabalho}
  Existem algumas formas diferentes de iniciar uma sessão de trabalho no R:
  \begin{itemize}
    \item Clicando no ícone na Área de trabalho (Win/Mac).
    \item Digitando ``R'' no terminal (Linux/Mac).
    \item Abrindo arquivos com extensões associadas ao R (.r, .RData, etc
).
  \end{itemize}
  \begin{center}
    \pgfuseimage{R}
  \end{center}
\end{frame}
 
\begin{frame}
  \frametitle{Encerrando uma sessão de trabalho}
  \begin{block}{Encerra uma sessão}
    q()
  \end{block}
\texttt{> q()\\
Save workspace image? [y/n/c]:}
\vspace{10pt}\\
Ao encerrar uma sessão o R pergunta se você pretende salvar a área de trabalho. Se responder \texttt{``y''} o R salva uma imagem de toda a área de trabalho num arquivo oculto \textit{.RData} dentro do diretório de trabalho.
\end{frame}
\begin{frame}
  \frametitle{Trabalhando com a imagem da área de trabalho}
  \begin{block}{Salvando a área de trabalho}
    \# Salva a área de trabalho no arquivo oculto \textit{.Rdata} sem encerrar a sessão. \\
     save.image()\\
    \vspace{10pt}
    \#  Salva a área de trabalho com o nome que você escolher  sem encerrar a sessão.\\
     save.image(file=``meuworkspace.RData'')\\
 \vspace{10pt}
   \# Carrega um arquivo de área de trabalho. \\
    load(file=``meuworkspace.RData'')
  \end{block}
 É recomendável salvar o trabalho  regularmente.  
\end{frame}

\begin{frame}
  \frametitle{Diretório de trabalho}
  
    Os arquivos \textit{.RData} são salvos no diretório de trabalho do R. Isso pode gerar problemas pois as novas análises podem sobrescrever as antigas. Por isso:
    \begin{itemize}
      \item <1-3>Crie um diretório diferente para cada projeto.
      \item <2-3> Coloque todos os seus arquivos de dados, planilhas etc nesse diretório.
     \item<3> Direcione o R para usar o diretório que você criou como diretório de trabalho.
    \end{itemize}
\end{frame}
\begin{frame}
  \frametitle{Diretório de trabalho}
  \begin{block}{Diretório de trabalho}
    \# Retorna qual diretório de trabalho atual.\\
     getwd() \\
    \# Altera o diretório de trabalho.\\
     setwd()
  \end{block}
\end{frame}
\section{Recomendações}
\begin{frame}
  \frametitle{Recomendações para trabalhar com o R}
  \begin{itemize}
    \item Use um diretório para cada projeto.
      \item Salve a área de trabalho regularmente, \texttt{save.image()}.
      \item Escreva seus comandos em um arquivo de texto.
      \item Escreva de forma organizada.
      \item Salve este arquivo no diretório de trabalho com extensão \texttt{.r} ou \texttt{.R}.
      \item Envie os comandos ao R:
        \begin{itemize}
         \item Usando o comando \texttt{source(``arquivo.r'')}. Executa todos os comandos de uma vez.
         \item Na interface R-GUI do Windows, use o editor de scripts e \texttt{ctrl+R} para enviar linhas ou blocos.
         \item \textbf{ Use editores próprios para programação em R}.
        \end{itemize}
  \end{itemize}
\end{frame}
\begin{frame}
  \frametitle{Editores próprios para R}
  
     Facilitam o trabalho de programação no R. Combinam ferramentas para escrever executar o código de modo mais eficiente. Experimente alguns até achar um que agrade.

     \begin{description}
     \item[Tinn-R] Editor de códigos brasileiro.\\
       \url{http://nbcgib.uesc.br/lec/software/editores/tinn-r/pt}\\
           \item[ESS] \textit{Emacs Speaks Statistics}  Extensão do Emacs para R.\\
             \url{http://ess.r-project.org/}\\
           \item[R studio] É uma das melhores opções para começar. \\
                 \url {http://www.rstudio.com/} 
     \end{description}
   \end{frame}
\begin{frame}
  \frametitle{Para a próxima aula}
  \begin{itemize}
    \item Realizar ``Uma hora de código''
  \item Instalar o R e o R-studio.
  \item Ler capitulo 1 e 2 da apostila.
  \item  Fazer os tutoriais ``Introdução ao R'' que estão em \url{http://ecologia.ib.usp.br/bie5782/doku.php?id=bie5782:02_tutoriais:start}
  \end{itemize}
  
\end{frame}




\end{document}