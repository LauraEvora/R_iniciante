\documentclass[handout]{beamer}
%\documentclass[handout,xcolor=pdftex,dvipsnames,table]{beamer}

%\usepackage[latin1]{inputenc}
%\documentclass[ucs]{beamer}%para sistemas com ucs
\usepackage[utf8]{inputenc}
\usepackage{verbatim}
\usepackage{tikz}   %TikZ is required for this to work.  Make sure this exists before the next line
\usepackage{pgf}

 \usepackage[scaled]{helvet}
 \renewcommand*\familydefault{\sfdefault} %% Only if the base font of the document is to be sans serif
 \usepackage[T1]{fontenc}

%\usepackage[utf8x]{inputenc}%idem
\usepackage[brazil]{babel}
\usepackage{verbatim}
\usetheme{Unicamp}%
\usecolortheme{dolphin}
%\usecolortheme{seahorse}
\usefonttheme[onlysmall]{structurebold}

\title[Linguagem R]{R para iniciantes\\ Aula VI Boas Práticas}
\author {Carlos Henrique Tonhatti}
\institute[Unicamp]{Universidade Estadual de Campinas}
\date{}



 \AtBeginSection[]
 {
\begin{frame}
\frametitle{Sumário}
\tableofcontents[currentsection]
\end{frame}
}

\AtBeginSubsection[]
{
   \begin{frame}
       \frametitle{Sumário}

       \tableofcontents[currentsection,currentsubsection]
       \setcounter{tocdepth}{2}
   \end{frame}
}


\begin{document}
%para criar a pagina de rosto
\frame{\titlepage} %inclui a front page 

%==================================================slide
\begin{frame}
  \frametitle{Dúvidas da última Aula?}
\end{frame}

% cria o sumario
\begin{frame}
 \frametitle{Sumário}
 \tableofcontents[pausesections]
  \setcounter{tocdepth}{2}% profundidade do sumario 
\end{frame}
%-------------------------------------------------------
%==================================================slide


\section{Boas práticas }

\begin{frame}{Boas práticas para lidar com projetos de dados}

\begin{itemize}
\item Comece cada programa com  a descrição sobre o que ele faz
  
\item Carregue todos os pacotes necessários logo no inicio
  
\item Considere que você está no diretório de trabalho.
  
\item  Comente as sessões de seu código
  
\item Coloque as definições  de funções no inicio do arquivo, ou em um arquivo separado
  
\item Use nomes e estilo de forma consistente
\item Quebre o código em pedaços pequenos 
  
\item Não altere os dados brutos

\item Sempre inicie um ambiente limpo ao invés de salvar o workspace (!)
  
\item Mantenha o registro das sessões
  
\item Tenha alguém para rever o código
  
\item Use controle de versão

  \end{itemize}
\end{frame}

\begin{frame}{Princípios ao lidar com análise de dados}

  \begin{itemize}
    
  \item Transparência: Organização lógica das unidades.
  \item Manutenção: Padronização e comentários objetivos.
    
  \item Modularidade: Separe em unidades lógicas.
    
  \item Portabilidade: use caminhos relativos, minimize dependências, deixe a dependências claras.
    
  \item Reprodutibilidade: Facilidade em repetir os resultados

    
  \item Eficiência para manter e modificar 
    \end{itemize}

   
\end{frame}




  \begin{frame}{Anatomia da pasta de trabalho}

    Use nomes de diretórios/ pastas de forma padrão para guardar os arquivos

    \begin{description}
      
    \item[Dados brutos] Pasta com os dados mais simples, menos manipulado
    \item[Dados processados] Dados limpos, transformados pronto para analise
      
    \item[Gráficos] Guardar os gráficos
      
    \item[doc] Documentação sobre o projeto, sobre os dados, artigos e rascunhos etc 
    \item[Relatórios] Relatórios gerados sobre os dados   
    \end{description}

    Além disso, vale a pena ter um arquivo Leiame.txt como um pequeno resumo do projeto. E um arquivo ParaFazer.txt.
    
    \end{frame}

\section{Controle de versão}
    \begin{frame}{Controle de Versão}

        
      \begin{center}
        Muito mais que um simples backup em nuvem 
  \pgfdeclareimage[width=6cm]{github}{github}
  \pgfuseimage{github}
\end{center}

      

\end{frame}


\begin{frame}{Mais sobre o GitHub}
  \begin{itemize}
  \item Você pode escolher e colocar o repo como privado. Há licença educacional para isso.
    
  \item Instalar no windows \url{https://dicasdeprogramacao.com.br/como-instalar-o-git-no-windows/}


    
  \item Um pequeno tutorial em PT-BR \url{https://rogerdudler.github.io/git-guide/index.pt_BR.html}

    
  \item O livro \url{https://git-scm.com/book/pt-br/v2}

    \end{itemize}

  \end{frame}



  \section{Rmarkdown}

  \begin{frame}{Rmarkdown}

    Uma linguagem de marcação de texto que gera relatórios dinâmicos. Facilita o trabalho de comunicação de resultados e aumenta a transparência da análise.

    
  \end{frame}


  \section{\dots e continua}
\begin{frame}{Assuntos que não foram cobertos pelo curso}
 Assuntos que não foram cobertos pelo curso e podem ser necessários dependendo do projeto.   
\begin{itemize}[<+->]
  \item Estatística (geral);
  \item Matemática (geral);
  \item Iteração e recursão (otimização, solução de problemas);
  \item Manipulação de palavras ``strings'' e expressão regular ( trabalhando com texto);
  \item Interação com outras linguagens (geral);
  \item Banco de dados (trabalhando com muitos dados);
  \item Computação paralela (otimização).
  \end{itemize}
  
\end{frame}

\begin{frame}{Pacotes }
  \begin{itemize}
  \item dplyr, tidyr, readr, ggplot2 tidyverse
  \item stringr, stringi, ``regex''
  \item popgenreport, ape, ade4, pegas hierfstat
  \item vegan, MASS, caret    
  \item doParallel
    
  \item roxigen2
    \end{itemize}

  \end{frame}


  \section{Trabalho final}
\begin{frame}{Trabalho final}
  O trabalho final da disciplina tem o objetivo de desenvolver programas que realizem análises mais completas que as vistas durante o curso. Cada trabalho dá ao aluno a chance de aprender algo novo. \\


  
  A seguir algumas propostas de Projeto, se tiver alguma outra ideia pode falar 

\end{frame}

\begin{frame}{Proposta 1}{Reconstrução filogenética}
\textbf{Reconstrução filogenética usando sequências do GenBank}\\
   Fonte:  Analysis of Phylogenetics and Evolution with R.\\
           Emmanuel Paradis. páginas:46--50, 121--125 ``Caso Sylvia''
           \begin{center}
             \pgfdeclareimage[width=6cm]{sylvia}{sylvia}
             \pgfuseimage{sylvia}
             \end{center}
 
\end{frame}
\begin{frame}{Proposta 2}{Modelagem de crescimento}
\textbf{Crescimento independente de densidade} \\
    Fonte : A primer of ecology with R.\\
          M. Henry H. Stevens. Cap. 1. Problema 1.1 
   
\end{frame}

\begin{frame}{Proposta 3}{Seleção de modelos}
\textbf{Seleção de modelos por verossimilhança}\\
   Fonte: \url{http://cmq.esalq.usp.br/BIE5781/doku.php?id=07-selecao:07-selecao}  
\end{frame}


\begin{frame}{Proposta 4}{Análise Exploratória de dados}


\begin{itemize}
\item Compreender como são os dados. Estatísticas de sumário. Como estão distribuídos, há padrões?
\item dados \url{https://archive.ics.uci.edu/ml/datasets/student+performance}

  
\item  Se quiser tentar uma PCA \url{https://www.r-bloggers.com/computing-and-visualizing-pca-in-r/}
\end{itemize}
\end{frame}

\begin{frame}{Proposta 5}
  \textbf{Bibliometria}\\

  \begin{itemize}
  \item Usar o pacote bibliometrix \url{https://cran.r-project.org/web/packages/bibliometrix/vignettes/bibliometrix-vignette.html}
    
  \item Usar uma palavra-chave da tua areá
  \item Criar gráficos: quem produz mais, mais citações, co-citação 
  \item Gerar rede de co-citação.
    \end{itemize}

  
  \end{frame}

\begin{frame}{Entrega}
  \begin{itemize}
  \item 21 dias para entregar por email
  \item Em pdf (com Rmarkdown) com todos os arquivos necessários
   
  \item Repositório do Github ou pasta zipada

  \item Roteiro
    \begin{itemize}
      
    \item Introdução (1 paragrafo)
    \item Requerimentos (pacotes)
      
    \item Desenvolvimento
    \item Respostas encontradas (1 paragrafo)
      
    \item Dificuldades encontradas
      
    \item Bibliografia (sem formato definido)
    \end{itemize}
    Pode acrescentar itens se quiser.
    \end{itemize}
  \end{frame}

  \begin{frame}{O que se espera}
    \begin{enumerate}
    \item  Dentro do prazo.
    \item Uso de Rmarkdown/PDF.
    \item Dentro do roteiro
      
    \item No PDF, todas as linhas de código de R aparecerem sem cortes
    \item Estilo de escrita do código, identação, clareza.
      
    \item Comentários no código, se criar alguma função comentar o que faz, qual entrada e qual a saída
    \item Github ou zipado com todos os arquivos necessários.
      
    \item Reprodutibilidade, o arquivo .rmd gera o PDF apresentado?
      
    \item Texto do documento.
      
    \end{enumerate}

    Nota: Cada item vale um ponto. Todas as atividades do Swirl somam um ponto a média. 
  \end{frame}
  
\end{document}